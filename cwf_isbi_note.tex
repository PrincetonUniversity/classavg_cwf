

\documentclass[11pt]{article} % use larger type; default would be 10pt

\usepackage[utf8]{inputenc} % set input encoding (not needed with XeLaTeX)
\usepackage{relsize}
\usepackage{amsmath}
%\usepackage{amssymb}
%sepackage{amsthm}
%%% Examples of Article customizations
% These packages are optional, depending whether you want the features they provide.
% See the LaTeX Companion or other references for full information.

%%% PAGE DIMENSIONS
\usepackage{geometry} % to change the page dimensions
\geometry{a4paper} % or letterpaper (US) or a5paper or....
% \geometry{margin=2in} % for example, change the margins to 2 inches all round
% \geometry{landscape} % set up the page for landscape
%   read geometry.pdf for detailed page layout information

\usepackage{graphicx} % support the \includegraphics command and options

% \usepackage[parfill]{parskip} % Activate to begin paragraphs with an empty line rather than an indent

%%% PACKAGES
\usepackage{booktabs} % for much better looking tables
\usepackage{array} % for better arrays (eg matrices) in maths
\usepackage{paralist} % very flexible & customisable lists (eg. enumerate/itemize, etc.)
\usepackage{verbatim} % adds environment for commenting out blocks of text & for better verbatim
\usepackage{subfig} % make it possible to include more than one captioned figure/table in a single float
% These packages are all incorporated in the memoir class to one degree or another...

%%% HEADERS & FOOTERS
\usepackage{fancyhdr} % This should be set AFTER setting up the page geometry
\pagestyle{fancy} % options: empty , plain , fancy
\renewcommand{\headrulewidth}{0pt} % customise the layout...
\lhead{}\chead{}\rhead{}
\lfoot{}\cfoot{\thepage}\rfoot{}

%%% SECTION TITLE APPEARANCE
\usepackage{sectsty}
\allsectionsfont{\sffamily\mdseries\upshape} % (See the fntguide.pdf for font help)
% (This matches ConTeXt defaults)

%%% ToC (table of contents) APPEARANCE
\usepackage[nottoc,notlof,notlot]{tocbibind} % Put the bibliography in the ToC
\usepackage[titles,subfigure]{tocloft} % Alter the style of the Table of Contents
\renewcommand{\cftsecfont}{\rmfamily\mdseries\upshape}
\renewcommand{\cftsecpagefont}{\rmfamily\mdseries\upshape} % No bold!

%%% END Article customizations

%%% The "real" document content comes below...

\title{Improving Class Averaging Using CWF}
\author{Tejal Bhamre}
%\date{} % Activate to display a given date or no date (if empty),
         % otherwise the current date is printed 
\date{\today}

\begin{document}
\maketitle

\section{Definitions}

$\begin{array}{l} Y_1=H_1 X_1 + N_1 \\ Y_2=H_2 X_2 + N_2 \\ X_1, X_2  \sim N( {\mu},\Sigma) \\ N_1, N_2  \sim N(0,{\sigma}^2 I_n )\end{array}$

\vspace{4 mm}

\noindent
$Y_1, Y_2$ are independent Gaussian random vectors.


\vspace{4 mm}

$\begin{array}{l} E(Y_1)=H_1 \mu \\ Cov(Y_1) = H_1 \Sigma {H_1}^T + {\sigma}^2 I_n = K_1 \\ E(Y_2)=H_2 \mu \\ Cov(Y_2) = H_2 \Sigma {H_2}^T + {\sigma}^2 I_n = K_2 \end{array}$

\vspace{4 mm}
\noindent
The pdf's are thus

\begin{equation}
f_{X_1}(x_1) = \frac{1}{({2\pi})^{\frac{n}{2}} {|\Sigma|}^{\frac{1}{2}}} \exp\{-\frac{1}{2}{(x_1-\mu)}^T {\Sigma}^{-1}(x_1-\mu)\} 
\end{equation}

\begin{equation}
f_{X_2}(x_2) = \frac{1}{({2\pi})^{\frac{n}{2}} {|\Sigma|}^{\frac{1}{2}}} \exp\{-\frac{1}{2}{(x_2-\mu)}^T {\Sigma}^{-1}(x_2-\mu)\} 
\end{equation}

\begin{equation}
f_N(y_1-H_1 x_1) = \frac{1}{({2\pi})^{\frac{n}{2}}\sigma^n} \exp\{-\frac{1}{2}{(y_1-H_1 x_1)}^T \frac{1}{\sigma^2}(y_1-H_1 x_1)\}
\end{equation}

\begin{equation}
f_N(y_2-H_2 x_2) = \frac{1}{({2\pi})^{\frac{n}{2}}\sigma^n} \exp\{-\frac{1}{2}{(y_2-H_2 x_2)}^T \frac{1}{\sigma^2} (y_2-H_2 x_2)\}
\end{equation}

\begin{equation}
f_{Y_1}(y_1) = \frac{1}{({2\pi})^{\frac{n}{2}} {|K_1|}^{\frac{1}{2}}} \exp\{-\frac{1}{2}{(y_1-H_1\mu)}^T {K_1}^{-1}(y_1-H_1\mu)\}
\end{equation}

\begin{equation}
f_{Y_2}(y_2) = \frac{1}{({2\pi})^{\frac{n}{2}} {|K_2|}^{\frac{1}{2}}} \exp\{-\frac{1}{2}{(y_2-H_2\mu)}^T {K_2}^{-1}(y_2-H_2\mu)\}
\end{equation}





\section{ Calculation}

\begin{equation}
 \left[\begin{array}{c} X_1 \\ Y_1 \end{array}\right] = 
\begin{bmatrix} I & 0 \\ H_1 & I \end{bmatrix} \times \left[ \begin{array}{c} X_1 \\ N_1 \end{array} \right]      
\end{equation}

\begin{equation}
 \sim N \left[\begin{bmatrix} \mu \\ H_1 \mu \end{bmatrix}, \begin{bmatrix} \Sigma & \Sigma H_1^T \\ H_1 \Sigma & H_1 \Sigma H_1^T + \sigma^2 I \end{bmatrix} \right]
\end{equation}

\noindent
Using conditional distributions (Wiki MND)
 \begin{equation}
  f_{X_1|Y_1}(x_1|y_1) \sim N(\alpha, L)
 \end{equation}
\begin{equation}
  f_{X_2|Y_2}(x_2|y_2) \sim N(\beta, M)
 \end{equation}
where 

$\begin{array}{l}\alpha = \mu + \Sigma H_1^T (H_1 \Sigma H_1^T + \sigma^2 I)^{-1} (y_1 - H_1 \mu) \\ L = \Sigma - \Sigma H_1^T (H_1 \Sigma H_1^T + \sigma^2 I)^{-1} H_1 \Sigma \\ \beta = \mu + \Sigma H_2^T (H_2 \Sigma H_2^T + \sigma^2 I)^{-1} (y_2 - H_2 \mu) \\ M = \Sigma - \Sigma H_2^T (H_2 \Sigma H_2^T + \sigma^2 I)^{-1} H_2 \Sigma        \end{array}$
         
So

\begin{equation}
P(x_1 - x_2|y_1, y_2) \sim N(\alpha-\beta, L+M)
\end{equation}

Let $X_1 - X_2 = X_3$. Then
\begin{equation}
  P(||x_3||_{\infty} < \epsilon|y_1,y_2)   =  \int_{-\epsilon }^{\epsilon} \frac{1}{(2 \pi)^{\frac{n}{2}} |L+M|^ \frac{1}{2}} \exp \{ -\frac{1}{2}(x_3 - (\alpha -\beta)^T)(L+M)^{-1}(x_3 - (\alpha -\beta)\}dx_3 
\end{equation}
 For small $\epsilon$ this is

\begin{equation}
= \frac{(2 \epsilon)^n}{(2 \pi)^{\frac{n}{2}} |L + M|^{\frac{1}{2}}} \exp\{-\frac{1}{2}(\alpha - \beta)^T(L+M)^{-1}(\alpha -\beta)\} 
\end{equation}

So we can define our metric after taking log and dropping out the constant term
\begin{equation}
 -\frac{1}{2}\log(|L + M|) -\frac{1}{2}(\alpha - \beta)^T(L+M)^{-1}(\alpha -\beta)
\end{equation}

\section{Implementation}

Note that $\alpha$ and $\beta$ defined above are denoised images obtained from CWF after deconvolution. In practice, we use $\alpha$, $\beta$, $L$, $M$ projected onto the subspace spanned by the principal components (of the respective angular frequency block).

We obtain an initial list of nearest neighbors for eacg image using the Initial Alignment procedure in ASPIRE (steerable PCA + bispectum). This list is then refined using the metric defined above.

We test this method on simulated projection images and observe an improvement in the nearest neighbors detected. 

\end{document} 
